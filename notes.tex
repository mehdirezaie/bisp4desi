Molino contains 75,000 galaxy mock catalogs designed to quantify the information content of any cosmological observable for a redshift-space galaxy sample. The galaxy catalogs are constructed from the Quijote N-body simulations (Villaescusa-Navarro et al. 2020) using the standard Zheng et al. (2007) Halo Occupation Distribution (HOD) model. The suite contains the galaxy catalogs necessary to conduct a Fisher matrix forecast over a full set of cosmological and HOD parameters for any cosmological observable for a redshift-space galaxy sample:

15,000 mocks at the fiducial cosmological and HOD parameters for covariance matrix estimation
60,000 mocks at 24 other parameter values (2,500 mocks each) to estimate the derivative of observables with respect to all six cosmological parameters (Omega_m, Omega_b, h, n_s, sigma_8, and M_nu) and five HOD parameters (logMmin, sigma_logM, logM0, alpha, and logM1).
The fiducial cosmological parameter values are Omega_m=0.3175, Omega_b=0.049, h=0.6711, n_s=0.9624, sigma_8=0.834, and M_nu = 0.0 eV. The fiducial HOD parameters are based on the high luminosity samples of the Sloan Digital Sky Survey (M_r < -21.5 and M_r < 22 samples): log Mmin=13.65, sigma_logM=0.2, log M0=14.0, alpha=1.1, log M_0=14.0. The other 24 non-fiducial parameter values are set by adjusting a single parameter either a step above or below the fiducial value. Since M_nu cannot be negative, the M_\nu steps are at 0.1, 0.2, 0.4 eV. At the fiducial parameters, the galaxy catalogs have number density n_g ~ 1.63x10^-4 h^3 Mpc^-3. Each galaxy catalog has a volume (1 Gpc/h)^3 and contains ~150,000 galaxies. Altogether, the Molino suite contains over 10 billion galaxies.