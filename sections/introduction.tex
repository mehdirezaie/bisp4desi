\section{Introduction}
\label{sec:introduction}
As a stage IV galaxy redshift survey, the Dark Energy Spectroscopic Instrument (DESI) utilizes $5000$ robotically-driven fibers to simultaneously collect spectra of thousands of galaxies and quasars \citep{aghamousa2016desi}. Over its five-year mission, DESI will yield an unparalleled volume of spectra to construct the biggest 3D volume of the Universe, which can be leveraged to deepen our understanding of the formation and evolution of structures in the Universe, and most importantly, to shed light on the nature of dark energy \citep{2016arXiv161100036D}.

Common clustering statistics for extracting information from data of galaxy redshift surveys, like DESI, are the power spectrum and two-point correlation function, which characterize the excess probability of finding two galaxies separated by a given wavenumber and distance, respectively, relative to a random distribution of galaxies. One of the primary features imprinted in the large-scale clustering of galaxies is Baryon Acoustic Oscillations (BAO), which are created by sound waves propagating through the early Universe before the epoch of recombination, i.e. when photons no longer were able to re-ionize atoms. These acoustic waves leave a characteristic scale in the distribution of galaxies that can be used as a standard ruler to measure distances in the Universe \citep[see, e.g.,][]{1996ApJ...471..542H, 1998ApJ...496..605E, 2003ApJ...598..720S}.
 
In galaxy redshift surveys, the position of galaxies is measured via their redshifts, which is not a direct measurement of their true distances, but rather influenced by their peculiar velocities, i.e., the velocities of the galaxies relative to their local environment. These peculiar velocities can smear out the BAO feature by causing the distribution of galaxies to appear elongated along the line-of-sight direction. Additionally, non-linearities caused by subsequent structure formation impact the sharpness of the BAO signal and effectively reduce the statistical precision of the distance scale measurements. Several reconstruction techniques have been developed to account for the smearing effect of BAO and recover the constraining power lost due to bulk flows and space distortions \citep[see, e.g.,][]{Eisenstein_2007}. However, even with reconstruction techniques applied, cosmic variance imposes a limitation on the amount of information that can be extracted from the two-point statistics of the large-scale structure. Cosmic variance can be reduced by increasing the size of the survey volume via either surveying a larger area of the sky or by going deeper in redshift. 

Measuring the correlation between triplets (or even quadruplets) of galaxies is an alternative way of accessing extra information that is independent of the one in the two-point statistics. The analysis of higher-order statistics from real data has so far resulted in moderate improvements in the cosmological constraints. The high-density DESI galaxy samples may perform significantly better. The standard ruler-based probes (such as the BAO) could, at least in principle, perform better than the reconstructed fields \cite{samushia2021information}.

Recent developments in algorithms utilizing rotational symmetries and advancements in computing power have made higher-order statistics more feasible to measure \citep[see, e.g.,][]{Szapudi_2004,2015MNRAS.454.4142S,2016MNRAS.455L..31S, 2017MNRAS.468.1070S, 2017MNRAS.469.2059S, 2018MNRAS.478.1468S, 2022PNAS..11911366P}. Higher order clustering measurements have been used to improve parameter constraints or break parameter degeneracies on modified theories of gravity \citep{2023arXiv230206808S}, galaxy-halo connections \citep{2022MNRAS.515.6133Z}, cosmic distance scales \citep{2017MNRAS.469.1738S, 2018MNRAS.478.4500P}, and baryon-dark matter relative velocity \citep{2018MNRAS.474.2109S}.

DESI is expected to deliver extremely precise measurements of galaxy bispectrum from all its samples. To robustly analyze the data at the level of DESI statistical errors will require overcoming multiple theoretical challenges. The precision matrices for the bispectrum are difficult to accurately estimate from the mock catalogs due to the large size of the data vector. The perturbation-theory-based methods for the bispectrum have been shown to break down on relatively large scales. 

In this work, we explore the possibility of measuring and analyzing the BAO feature in the average (angular monopole) bispectrum. Even though the BAO feature extends to a higher wavenumber, the beat frequency is a large-scale feature set by linear physics in the early Universe. There is a hope that the extracted BAO frequency can be made unbiased even when the models for the full shape of the galaxy bispectrum fail to predict it with the desired accuracy.

Not having accurate covariance matrices for our measurements hinders our ability to directly interpret our likelihood surfaces. We use a number of indirect methods to assess the variance in the measurement of the BAO scale from the bispectrum compared to the power spectrum. Some of these methods rely on artificially constructed covariance matrices that are conservative estimates of the true covariance. Other methods estimate the variance by looking at the spread of BAO scale measurements over multiple mock catalogs.

Our preliminary findings are encouraging. The BAO feature is clearly visible in the measured bispectrum. A significant fraction of this signal does not seem to be strongly correlated with the same signal in the power spectrum. \textcolor{red}{We will put here some specific numbers once we have stable results}.

In this paper, we utilize simulations of galaxies with similar clustering properties as the DESI targets to characterize the constraining power on cosmic distance scales from the bispectrum. The outline of this paper is as follows. We discuss the templates for BAO power spectrum and bispectrum in Section \ref{sec:templates}, and the details of simulations in Section \ref{sec:measurements}. Finally, we present the constraints in \ref{sec:fits}, and conclude with discussion and conclusion in \ref{sec:conclusion}.