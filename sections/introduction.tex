\section{Introduction}
\label{sec:introduction}

DESI is a great experiment. We need higher order statistics to extract more information. Reconstruction techniques can improve the precision, but the amount of information is limited by cosmic variance. Possible ways to improve is to go higher order statistics. 

Cite Lado's paper that if bispectrum used as a standard ruler, a significant improvement is distance scale can be achieved. Much better than the standard reconstruction. Difficult to model the full bispectrum though. In this paper, we would like to do a standard ruler analysis of bispectrum BAO with different DESI-like tracers. 

We find a factor of X improvement, and what happens at different redshifts. Worse for QSO probably, because of high shotnoise and less bispectrum clustering at higher redshifts. 


Sec 2. Glam Section
Plot Glam with and without BAO measurements.
Describe Molino's covariance and put some justification. 1 Gpc cub.
Reference Jayashree's paper that the template works well for a range of redshifts. make a plot for sigma alpha vs kmax, with and without nuisance parameters.
Use CAMB to get the bao constraints for reconstructed power spectrum with GLAM parameters.
Sec 3. BAO is DESI samples. Analyse Abacus mocks with template from Jayashree. Renormalize Molino's covariance by the ratio of spectra to get covariance for Abacus. Maybe use Glam and Molino to show that spectra ratio scales proportional to dispersion ratio.

Sec 4. BAO detection level. How well we can fit BAO with a smooth function. Mean Glam' with and without BAO as model 1 and 2, and mean Glam with BAO as data. Chi2 vs alpha.



\begin{equation}
\log \mathcal{L} = \Delta r^{\dagger} C^{-1} \Delta r,
\end{equation}
where $\Delta r = r(\textbf{k}) - r(\alpha \textbf{k})$ with $r$ being defined as the ratio of




